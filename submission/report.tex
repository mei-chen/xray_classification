\documentclass[11pt,a4paper]{article}

% Packages
\usepackage[utf8]{inputenc}
\usepackage[T1]{fontenc}
\usepackage{lmodern}
\usepackage[margin=1in]{geometry}
\usepackage{graphicx}
\usepackage{booktabs}
\usepackage{hyperref}
\usepackage{xcolor}
\usepackage{float}
\usepackage{amsmath}
\usepackage{enumitem}
\usepackage{fancyhdr}
\usepackage{titlesec}

% Colors
\definecolor{primary}{RGB}{41, 128, 185}
\definecolor{accent}{RGB}{231, 76, 60}
\definecolor{success}{RGB}{39, 174, 96}

% Hyperref setup
\hypersetup{
    colorlinks=true,
    linkcolor=primary,
    urlcolor=primary,
    citecolor=primary
}

% Header/Footer
\pagestyle{fancy}
\fancyhf{}
\rhead{Chest X-Ray Classification}
\lhead{Technical Report}
\rfoot{Page \thepage}

% Title formatting
\titleformat{\section}{\Large\bfseries\color{primary}}{\thesection}{1em}{}
\titleformat{\subsection}{\large\bfseries}{\thesubsection}{1em}{}

% Document
\begin{document}

% Title
\begin{center}
    {\LARGE\bfseries\color{primary} Chest X-Ray Classification System}\\[0.5em]
    {\large Technical Report}\\[1em]
    {\normalsize Deep Learning for Medical Image Diagnosis}\\[0.5em]
    \today
\end{center}

\vspace{1em}

%==============================================================================
\section{Executive Summary}
%==============================================================================

This report presents a deep learning system for classifying chest X-ray images into three diagnostic categories: \textbf{Normal}, \textbf{Pneumonia}, and \textbf{Tuberculosis}. The system achieves \textbf{90.8\% macro AUC} on a held-out test set of 2,569 images, with particularly strong performance on Pneumonia detection (98.7\% AUC, 100\% recall).

\begin{table}[H]
\centering
\begin{tabular}{lc}
\toprule
\textbf{Metric} & \textbf{Score} \\
\midrule
Overall Accuracy & 77.1\% \\
Macro AUC-ROC & 90.8\% \\
Test Samples & 2,569 \\
\bottomrule
\end{tabular}
\caption{Summary of model performance on test set.}
\end{table}

%==============================================================================
\section{EDA Findings and Modeling Choices}
%==============================================================================

\subsection{Dataset Characteristics}

The dataset consists of chest X-ray images from three classes with notable class imbalance:

\begin{table}[H]
\centering
\begin{tabular}{lccc}
\toprule
\textbf{Class} & \textbf{Train} & \textbf{Test} & \textbf{Proportion} \\
\midrule
Normal & -- & 925 & 36.0\% \\
Pneumonia & -- & 580 & 22.6\% \\
Tuberculosis & -- & 1,064 & 41.4\% \\
\bottomrule
\end{tabular}
\caption{Class distribution in test set (imbalance ratio $\approx$ 1.8$\times$).}
\end{table}

\subsection{Key EDA Findings}

\begin{enumerate}[leftmargin=*]
    \item \textbf{Class Imbalance}: Tuberculosis samples are nearly 2$\times$ more common than Pneumonia, requiring weighted loss functions and balanced sampling.
    
    \item \textbf{Image Properties}: Variable image sizes and aspect ratios; most images are grayscale requiring RGB conversion for transfer learning.
    
    \item \textbf{Visual Patterns}: Pneumonia shows diffuse bilateral infiltrates; Tuberculosis often shows upper lobe consolidation and cavitation; Normal cases show clear lung fields.
\end{enumerate}

\subsection{Modeling Choices}

Based on EDA findings, the following design decisions were made:

\begin{table}[H]
\centering
\begin{tabular}{ll}
\toprule
\textbf{Decision} & \textbf{Rationale} \\
\midrule
EfficientNet-B0 backbone & Best accuracy/efficiency trade-off \\
Class-weighted loss & Address 1.8$\times$ imbalance \\
Weighted random sampling & Balance mini-batches \\
Label smoothing (0.1) & Reduce overconfidence \\
Progressive unfreezing & Better transfer learning \\
Data augmentation & Increase effective dataset size \\
\bottomrule
\end{tabular}
\caption{Key modeling decisions and their rationale.}
\end{table}

%==============================================================================
\section{Final Metrics and Calibration}
%==============================================================================

\subsection{Per-Class Performance}

\begin{table}[H]
\centering
\begin{tabular}{lccccc}
\toprule
\textbf{Class} & \textbf{Precision} & \textbf{Recall} & \textbf{F1} & \textbf{AUC} & \textbf{Support} \\
\midrule
Normal & 64.8\% & 80.4\% & 71.8\% & 84.0\% & 925 \\
Pneumonia & 77.5\% & \textbf{100\%} & 87.3\% & \textbf{98.7\%} & 580 \\
Tuberculosis & \textbf{97.5\%} & 61.7\% & 75.5\% & 89.7\% & 1,064 \\
\midrule
\textbf{Macro Avg} & 79.9\% & 80.7\% & 78.2\% & \textbf{90.8\%} & 2,569 \\
\bottomrule
\end{tabular}
\caption{Detailed classification metrics by class.}
\end{table}

\subsection{Confusion Matrix Analysis}

\begin{figure}[H]
\centering
\includegraphics[width=0.55\textwidth]{confusion_matrix.png}
\caption{Normalized confusion matrix showing prediction patterns.}
\end{figure}

\textbf{Key observations}:
\begin{itemize}[leftmargin=*]
    \item \textbf{Pneumonia}: Perfect recall (100\%)---no missed cases, critical for screening.
    \item \textbf{Tuberculosis}: High precision (97.5\%)---very few false positives.
    \item \textbf{Normal}: Some confusion with Pneumonia (18\%), expected given visual similarity.
    \item \textbf{TB$\rightarrow$Normal confusion}: 38\% of TB cases misclassified as Normal, indicating room for improvement.
\end{itemize}

\subsection{ROC Curves}

\begin{figure}[H]
\centering
\includegraphics[width=0.7\textwidth]{roc_curves.png}
\caption{One-vs-Rest ROC curves for each class.}
\end{figure}

The strong AUC scores (all $>$0.84) indicate the model learns discriminative features. Pneumonia's near-perfect AUC (0.987) suggests it has the most distinctive radiographic features.

\subsection{Calibration}

The model incorporates several calibration techniques:
\begin{itemize}[leftmargin=*]
    \item \textbf{Label smoothing} (0.1) prevents overconfident predictions
    \item \textbf{Temperature scaling} can be applied post-hoc for probability calibration
    \item Softmax outputs provide probability estimates for clinical decision support
\end{itemize}

%==============================================================================
\section{Robustness Analysis}
%==============================================================================

While formal robustness testing (Task 3) was not completed, the following robustness considerations were addressed:

\subsection{Training Robustness}
\begin{itemize}[leftmargin=*]
    \item \textbf{Data augmentation}: Random flips, rotations ($\pm$15°), brightness/contrast jitter, and affine transforms simulate real-world variation.
    \item \textbf{Dropout} (0.3): Prevents overfitting to training distribution.
    \item \textbf{Early stopping}: Prevents overfitting with patience of 10 epochs.
\end{itemize}

\subsection{Potential Vulnerabilities}
\begin{itemize}[leftmargin=*]
    \item \textbf{Domain shift}: Performance may degrade on X-rays from different scanners, hospitals, or patient demographics.
    \item \textbf{TB$\rightarrow$Normal misclassification}: 38\% of TB cases misclassified suggests sensitivity to subtle TB patterns.
    \item \textbf{Image quality}: Model assumes reasonable X-ray quality; heavily degraded images may fail.
\end{itemize}

\subsection{Recommended Future Testing}
\begin{enumerate}[leftmargin=*]
    \item Test on external datasets (e.g., NIH ChestX-ray14, CheXpert)
    \item Adversarial robustness evaluation
    \item Noise and blur perturbation testing
    \item Subgroup analysis by patient demographics (if available)
\end{enumerate}

%==============================================================================
\section{Explainability Insights}
%==============================================================================

\subsection{Grad-CAM Visualization}

Gradient-weighted Class Activation Mapping (Grad-CAM) was implemented to visualize which regions of the X-ray influence predictions.

\begin{figure}[H]
\centering
\includegraphics[width=\textwidth]{gradcam_analysis.png}
\caption{Grad-CAM attention maps for sample predictions. Bright regions indicate high importance.}
\end{figure}

\subsection{Clinical Alignment}

The attention maps reveal clinically meaningful patterns:

\begin{table}[H]
\centering
\begin{tabular}{lp{8cm}}
\toprule
\textbf{Class} & \textbf{Model Attention Focus} \\
\midrule
Normal & Clear lung fields, absence of pathology markers \\
Pneumonia & Diffuse infiltrates, bilateral opacities, lower/middle zones \\
Tuberculosis & Upper lobe consolidation, apical regions, cavitary patterns \\
\bottomrule
\end{tabular}
\caption{Grad-CAM attention patterns align with known radiographic findings.}
\end{table}

These patterns align with established radiological knowledge, increasing confidence in the model's decision-making process.

%==============================================================================
\section{Deployment Plan}
%==============================================================================

\subsection{API Architecture}

A production-ready REST API was implemented using FastAPI:

\begin{table}[H]
\centering
\begin{tabular}{llp{6cm}}
\toprule
\textbf{Endpoint} & \textbf{Method} & \textbf{Description} \\
\midrule
\texttt{/health} & GET & Service health check \\
\texttt{/model/info} & GET & Model metadata (architecture, classes) \\
\texttt{/predict} & POST & Single image classification \\
\texttt{/predict/batch} & POST & Batch classification ($\leq$10 images) \\
\texttt{/predict/explain} & POST & Classification + Grad-CAM visualization \\
\bottomrule
\end{tabular}
\caption{API endpoint summary.}
\end{table}

\subsection{Containerization}

Docker deployment with:
\begin{itemize}[leftmargin=*]
    \item Multi-stage build for minimal image size
    \item Non-root user for security
    \item Health check endpoint for orchestration
    \item CORS middleware for frontend integration
\end{itemize}

\subsection{Deployment Commands}

\begin{verbatim}
# Local development
uvicorn src.api.main:app --host 0.0.0.0 --port 8000

# Docker deployment
docker build -t xray-classifier .
docker run -p 8000:8000 xray-classifier
\end{verbatim}

\subsection{Clinical Integration Considerations}

\begin{enumerate}[leftmargin=*]
    \item \textbf{Decision support, not replacement}: Model outputs should assist, not replace, radiologist judgment.
    \item \textbf{Confidence thresholds}: Low-confidence predictions should be flagged for expert review.
    \item \textbf{Audit logging}: All predictions should be logged for retrospective analysis.
    \item \textbf{Regulatory}: FDA 510(k) clearance required for clinical use in the US.
\end{enumerate}

%==============================================================================
\section{Conclusion}
%==============================================================================

This project demonstrates a complete ML pipeline from data exploration to deployment-ready API. Key achievements:

\begin{itemize}[leftmargin=*]
    \item \textbf{Strong discriminative performance}: 90.8\% macro AUC across three classes
    \item \textbf{Perfect Pneumonia recall}: Zero missed cases in test set
    \item \textbf{Interpretable predictions}: Grad-CAM visualizations align with clinical knowledge
    \item \textbf{Production-ready}: Containerized API with explainability endpoints
\end{itemize}

\textbf{Limitations}: The main weakness is TB$\rightarrow$Normal confusion (38\%), which could be addressed with additional TB training data, class-specific augmentation, or ensemble methods.

\vspace{1em}
\hrule
\vspace{0.5em}
\small\textit{Built with PyTorch, timm, FastAPI, and best MLOps practices.}

\end{document}

